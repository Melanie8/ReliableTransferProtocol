\documentclass{article}

\usepackage[utf8x]{inputenc}
\usepackage[frenchb]{babel}
\usepackage[T1]{fontenc}
\usepackage{lmodern}
\usepackage{fullpage}
\usepackage{graphicx}
\usepackage{epstopdf}
\usepackage{caption}
\usepackage{subcaption}
\usepackage{multirow}

% Math symbols
\usepackage{amsmath}
\usepackage{amssymb}
\usepackage{amsthm}

% Numbers and units
\usepackage[squaren, Gray]{SIunits}
\usepackage{sistyle}
\usepackage[autolanguage]{numprint}
%\usepackage{numprint}
\newcommand\si[2]{\numprint[#2]{#1}}
\newcommand\np[1]{\numprint{#1}}

\DeclareMathOperator{\newdiff}{d} % use \dif instead
\newcommand{\dif}{\newdiff\!}
\newcommand{\fpart}[2]{\frac{\partial #1}{\partial #2}}
\newcommand{\ffpart}[2]{\frac{\partial^2 #1}{\partial #2^2}}
\newcommand{\fdpart}[3]{\frac{\partial^2 #1}{\partial #2\partial #3}}
\newcommand{\fdif}[2]{\frac{\dif #1}{\dif #2}}
\newcommand{\ffdif}[2]{\frac{\dif^2 #1}{\dif #2^2}}
\newcommand{\constant}{\ensuremath{\mathrm{cst}}}

% Listing
% Color
% cfr http://en.wikibooks.org/wiki/LaTeX/Colors
\usepackage{color}
\usepackage[usenames,dvipsnames,svgnames,table]{xcolor}
\definecolor{dkgreen}{rgb}{0.25,0.7,0.35}
\definecolor{dkred}{rgb}{0.7,0,0}


\usepackage{listings}
\lstset{
  numbers=left,
  numberstyle=\tiny\color{gray},
  basicstyle=\rm\small\ttfamily,
  keywordstyle=\bfseries\color{dkred},
  frame=single,
  commentstyle=\color{gray}=small,
  stringstyle=\color{dkgreen},
  %backgroundcolor=\color{gray!10},
  %tabsize=2,
  rulecolor=\color{black!30},
  %title=\lstname,
  breaklines=true,
  framextopmargin=2pt,
  framexbottommargin=2pt,
  extendedchars=true,
  inputencoding=utf8x,
  language=C
}


\author{Benoît Legat}
\title{RTP}

\begin{document}

\maketitle

\appendix

We need to choose family 10 (IPv6) and protocol 17 (UDP).

\section{Arguments}
\lstinline+/usr/include/bits/socket.h+
\begin{lstlisting}
/* Address families.  */
#define	AF_UNSPEC	PF_UNSPEC
#define	AF_LOCAL	PF_LOCAL
#define	AF_UNIX		PF_UNIX
#define	AF_FILE		PF_FILE
#define	AF_INET		PF_INET
#define	AF_AX25		PF_AX25
#define	AF_IPX		PF_IPX
#define	AF_APPLETALK	PF_APPLETALK
#define	AF_NETROM	PF_NETROM
#define	AF_BRIDGE	PF_BRIDGE
#define	AF_ATMPVC	PF_ATMPVC
#define	AF_X25		PF_X25
#define	AF_INET6	PF_INET6
#define	AF_ROSE		PF_ROSE
#define	AF_DECnet	PF_DECnet
#define	AF_NETBEUI	PF_NETBEUI
#define	AF_SECURITY	PF_SECURITY
#define	AF_KEY		PF_KEY
#define	AF_NETLINK	PF_NETLINK
#define	AF_ROUTE	PF_ROUTE
#define	AF_PACKET	PF_PACKET
#define	AF_ASH		PF_ASH
#define	AF_ECONET	PF_ECONET
#define	AF_ATMSVC	PF_ATMSVC
#define AF_RDS		PF_RDS
#define	AF_SNA		PF_SNA
#define	AF_IRDA		PF_IRDA
#define	AF_PPPOX	PF_PPPOX
#define	AF_WANPIPE	PF_WANPIPE
#define AF_LLC		PF_LLC
#define AF_CAN		PF_CAN
#define AF_TIPC		PF_TIPC
#define	AF_BLUETOOTH	PF_BLUETOOTH
#define	AF_IUCV		PF_IUCV
#define AF_RXRPC	PF_RXRPC
#define AF_ISDN		PF_ISDN
#define AF_PHONET	PF_PHONET
#define AF_IEEE802154	PF_IEEE802154
#define AF_CAIF		PF_CAIF
#define AF_ALG		PF_ALG
#define AF_NFC		PF_NFC
#define AF_VSOCK	PF_VSOCK
#define	AF_MAX		PF_MAX
\end{lstlisting}

\lstinline+/usr/include/bits/socket_type.h+
Only 1, 2 and 3 are returned when we set 0 with protocols 6, 17 and 0 respectively.
\begin{lstlisting}
/* Types of sockets.  */
enum __socket_type
{
  SOCK_STREAM = 1,		/* Sequenced, reliable, connection-based
				   byte streams.  */
#define SOCK_STREAM SOCK_STREAM
  SOCK_DGRAM = 2,		/* Connectionless, unreliable datagrams
				   of fixed maximum length.  */
#define SOCK_DGRAM SOCK_DGRAM
  SOCK_RAW = 3,			/* Raw protocol interface.  */
#define SOCK_RAW SOCK_RAW
  SOCK_RDM = 4,			/* Reliably-delivered messages.  */
#define SOCK_RDM SOCK_RDM
  SOCK_SEQPACKET = 5,		/* Sequenced, reliable, connection-based,
				   datagrams of fixed maximum length.  */
#define SOCK_SEQPACKET SOCK_SEQPACKET
  SOCK_DCCP = 6,		/* Datagram Congestion Control Protocol.  */
#define SOCK_DCCP SOCK_DCCP
  SOCK_PACKET = 10,		/* Linux specific way of getting packets
				   at the dev level.  For writing rarp and
				   other similar things on the user level. */
#define SOCK_PACKET SOCK_PACKET

  /* Flags to be ORed into the type parameter of socket and socketpair and
     used for the flags parameter of paccept.  */

  SOCK_CLOEXEC = 02000000,	/* Atomically set close-on-exec flag for the
				   new descriptor(s).  */
#define SOCK_CLOEXEC SOCK_CLOEXEC
  SOCK_NONBLOCK = 00004000	/* Atomically mark descriptor(s) as
				   non-blocking.  */
#define SOCK_NONBLOCK SOCK_NONBLOCK
};
\end{lstlisting}

\lstinline+/usr/include/bits/in.h+
\begin{lstlisting}
/* Standard well-defined IP protocols.  */
enum
  {
    IPPROTO_IP = 0,	   /* Dummy protocol for TCP.  */
#define IPPROTO_IP		IPPROTO_IP
    IPPROTO_ICMP = 1,	   /* Internet Control Message Protocol.  */
#define IPPROTO_ICMP		IPPROTO_ICMP
    IPPROTO_IGMP = 2,	   /* Internet Group Management Protocol. */
#define IPPROTO_IGMP		IPPROTO_IGMP
    IPPROTO_IPIP = 4,	   /* IPIP tunnels (older KA9Q tunnels use 94).  */
#define IPPROTO_IPIP		IPPROTO_IPIP
    IPPROTO_TCP = 6,	   /* Transmission Control Protocol.  */
#define IPPROTO_TCP		IPPROTO_TCP
    IPPROTO_EGP = 8,	   /* Exterior Gateway Protocol.  */
#define IPPROTO_EGP		IPPROTO_EGP
    IPPROTO_PUP = 12,	   /* PUP protocol.  */
#define IPPROTO_PUP		IPPROTO_PUP
    IPPROTO_UDP = 17,	   /* User Datagram Protocol.  */
#define IPPROTO_UDP		IPPROTO_UDP
    IPPROTO_IDP = 22,	   /* XNS IDP protocol.  */
#define IPPROTO_IDP		IPPROTO_IDP
    IPPROTO_TP = 29,	   /* SO Transport Protocol Class 4.  */
#define IPPROTO_TP		IPPROTO_TP
    IPPROTO_DCCP = 33,	   /* Datagram Congestion Control Protocol.  */
#define IPPROTO_DCCP		IPPROTO_DCCP
    IPPROTO_IPV6 = 41,     /* IPv6 header.  */
#define IPPROTO_IPV6		IPPROTO_IPV6
    IPPROTO_RSVP = 46,	   /* Reservation Protocol.  */
#define IPPROTO_RSVP		IPPROTO_RSVP
    IPPROTO_GRE = 47,	   /* General Routing Encapsulation.  */
#define IPPROTO_GRE		IPPROTO_GRE
    IPPROTO_ESP = 50,      /* encapsulating security payload.  */
#define IPPROTO_ESP		IPPROTO_ESP
    IPPROTO_AH = 51,       /* authentication header.  */
#define IPPROTO_AH		IPPROTO_AH
    IPPROTO_MTP = 92,	   /* Multicast Transport Protocol.  */
#define IPPROTO_MTP		IPPROTO_MTP
    IPPROTO_BEETPH = 94,   /* IP option pseudo header for BEET.  */
#define IPPROTO_BEETPH		IPPROTO_BEETPH
    IPPROTO_ENCAP = 98,	   /* Encapsulation Header.  */
#define IPPROTO_ENCAP		IPPROTO_ENCAP
    IPPROTO_PIM = 103,	   /* Protocol Independent Multicast.  */
#define IPPROTO_PIM		IPPROTO_PIM
    IPPROTO_COMP = 108,	   /* Compression Header Protocol.  */
#define IPPROTO_COMP		IPPROTO_COMP
    IPPROTO_SCTP = 132,	   /* Stream Control Transmission Protocol.  */
#define IPPROTO_SCTP		IPPROTO_SCTP
    IPPROTO_UDPLITE = 136, /* UDP-Lite protocol.  */
#define IPPROTO_UDPLITE		IPPROTO_UDPLITE
    IPPROTO_RAW = 255,	   /* Raw IP packets.  */
#define IPPROTO_RAW		IPPROTO_RAW
    IPPROTO_MAX
  };
\end{lstlisting}

\lstinputlisting[caption={\lstinline+/etc/protocols+}]{protocols}

\begin{lstlisting}
/* Possible values for `ai_flags' field in `addrinfo' structure.  */
# define AI_PASSIVE	0x0001	/* Socket address is intended for `bind'.  */
# define AI_CANONNAME	0x0002	/* Request for canonical name.  */
# define AI_NUMERICHOST	0x0004	/* Don't use name resolution.  */
# define AI_V4MAPPED	0x0008	/* IPv4 mapped addresses are acceptable.  */
# define AI_ALL		0x0010	/* Return IPv4 mapped and IPv6 addresses.  */
# define AI_ADDRCONFIG	0x0020	/* Use configuration of this host to choose
				   returned address type..  */
# ifdef __USE_GNU
#  define AI_IDN	0x0040	/* IDN encode input (assuming it is encoded
				   in the current locale's character set)
				   before looking it up. */
#  define AI_CANONIDN	0x0080	/* Translate canonical name from IDN format. */
#  define AI_IDN_ALLOW_UNASSIGNED 0x0100 /* Don't reject unassigned Unicode
					    code points.  */
#  define AI_IDN_USE_STD3_ASCII_RULES 0x0200 /* Validate strings according to
						STD3 rules.  */
# endif
# define AI_NUMERICSERV	0x0400	/* Don't use name resolution.  */

/* Error values for `getaddrinfo' function.  */
# define EAI_BADFLAGS	  -1	/* Invalid value for `ai_flags' field.  */
# define EAI_NONAME	  -2	/* NAME or SERVICE is unknown.  */
# define EAI_AGAIN	  -3	/* Temporary failure in name resolution.  */
# define EAI_FAIL	  -4	/* Non-recoverable failure in name res.  */
# define EAI_FAMILY	  -6	/* `ai_family' not supported.  */
# define EAI_SOCKTYPE	  -7	/* `ai_socktype' not supported.  */
# define EAI_SERVICE	  -8	/* SERVICE not supported for `ai_socktype'.  */
# define EAI_MEMORY	  -10	/* Memory allocation failure.  */
# define EAI_SYSTEM	  -11	/* System error returned in `errno'.  */
# define EAI_OVERFLOW	  -12	/* Argument buffer overflow.  */
# ifdef __USE_GNU
#  define EAI_NODATA	  -5	/* No address associated with NAME.  */
#  define EAI_ADDRFAMILY  -9	/* Address family for NAME not supported.  */
#  define EAI_INPROGRESS  -100	/* Processing request in progress.  */
#  define EAI_CANCELED	  -101	/* Request canceled.  */
#  define EAI_NOTCANCELED -102	/* Request not canceled.  */
#  define EAI_ALLDONE	  -103	/* All requests done.  */
#  define EAI_INTR	  -104	/* Interrupted by a signal.  */
#  define EAI_IDN_ENCODE  -105	/* IDN encoding failed.  */
# endif

# ifdef __USE_MISC
#  define NI_MAXHOST      1025
#  define NI_MAXSERV      32
# endif

# define NI_NUMERICHOST	1	/* Don't try to look up hostname.  */
# define NI_NUMERICSERV 2	/* Don't convert port number to name.  */
# define NI_NOFQDN	4	/* Only return nodename portion.  */
# define NI_NAMEREQD	8	/* Don't return numeric addresses.  */
# define NI_DGRAM	16	/* Look up UDP service rather than TCP.  */
# ifdef __USE_GNU
#  define NI_IDN	32	/* Convert name from IDN format.  */
#  define NI_IDN_ALLOW_UNASSIGNED 64 /* Don't reject unassigned Unicode
					code points.  */
#  define NI_IDN_USE_STD3_ASCII_RULES 128 /* Validate strings according to
					     STD3 rules.  */
# endif

/* Translate name of a service location and/or a service name to set of
   socket addresses.

   This function is a possible cancellation point and therefore not
   marked with __THROW.  */
extern int getaddrinfo (const char *__restrict __name,
			const char *__restrict __service,
			const struct addrinfo *__restrict __req,
			struct addrinfo **__restrict __pai);

/* Free `addrinfo' structure AI including associated storage.  */
extern void freeaddrinfo (struct addrinfo *__ai) __THROW;

/* Convert error return from getaddrinfo() to a string.  */
extern const char *gai_strerror (int __ecode) __THROW;
\end{lstlisting}

\end{document}
